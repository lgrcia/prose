{\fontfamily{cmss}\selectfont

\hspace{0.8cm}
\begin{tabular}{m{0.92\linewidth}}
\parbox{\linewidth}{

  \boxtitle{ADDITIONAL NOTES}
  \mbox{ \vbox{ \vspace{1cm}\textbf{Goal of the observation}: check the return of the transit at the revised ephemeris in a red filter. \\
  \begin{itemize}
      \item TOI-2363.01 has been observed with the TRAPPIST-South 60 cm telescope on the night of UTC 2021.03.06 in the z' filter.
      \item The transit is detected on the target star which is labeled 18 on the stack image and with a depth of about 6 ppt. 
      \item The transit timing is consistent with the TTF ephemeris, although a meridian flip took place close to the ingress making its determination challenging. 
      \item The target and comparison star light curves have been detrended for the meridian flip and the fwhm.
      \item Three stars are used for comparison, all of them flat and of similar brightness as the target. 
      \item Weather conditions were stable throughout the observation and the moon was at 36\% at 101°.
      \item Comments in TTF before the observation:
       QLP S8 PC: [P=5.54499] Crowded field; CTOI from Marco Montalto; Don Radford observed a full (+/-3?) on 20201208 in I and did not see a clear 6 ppt event on or off target, although a triple detrended light curve suggests a ~30 min late ~5 ppt transit on target. SAFFIs prefer an event on target, or possibly T2 or T3. TG Tan observed a nominal full on 20210218 in Rc and detected a 260 min (2.4?) early ~7ppt egress in an uncontaminated 7.4" aperture. [P=5.5436040 - same as QLP s33] The next observation should be a full transit in a red (r', R, i', I, z) filter to check for the return of the transit at the revised ephemeris. Multi-band observations preferred.
  \end{itemize}
   \textbf{Conclusions}: The TRAPPIST team observed an on time full transit ~6 ppt deep in the z' filter.
   }}
} 
\end{tabular}
}